\documentclass{article}
\usepackage{amsmath}
\usepackage{amssymb}

\begin{document}

\section{Altitude Data Processing with \texttt{altitude\_data.py}}
This document provides an overview of how the \texttt{altitude\_data.py} script processes altitude data for the Balloon project.

\subsection{Overview}
The \texttt{altitude\_data.py} script is designed to read, process, and analyze altitude data stored in a text file (\texttt{altitude\_data.txt}). It is a critical component of the Balloon project, enabling the evaluation of flight performance and environmental conditions.

\subsection{Data Input}
The script reads altitude data from a file named \texttt{altitude\_data.txt}. The data is expected to be in a structured format, with each line representing a single altitude measurement in meters. For example:
\begin{verbatim}
100
150
200
250
\end{verbatim}

\subsection{Data Processing}
The script processes the altitude data using the following steps:
\begin{enumerate}
    \item \textbf{File Reading:} The script opens \texttt{altitude\_data.txt} in read mode and loads the data into memory.
    \item \textbf{Data Cleaning:} Any invalid or non-numeric entries are filtered out to ensure the integrity of the dataset.
    \item \textbf{Statistical Analysis:} The script calculates key metrics such as:
    \begin{itemize}
        \item \textbf{Mean Altitude:} The average altitude across all measurements.
        \item \textbf{Maximum Altitude:} The highest altitude recorded.
        \item \textbf{Minimum Altitude:} The lowest altitude recorded.
    \end{itemize}
    \item \textbf{Visualization:} The script generates plots to visualize altitude trends over time, if applicable.
\end{enumerate}

\subsection{Key Functions}
The following are the primary functions implemented in \texttt{altitude\_data.py}:
\begin{itemize}
    \item \texttt{read\_data(file\_path):} Reads altitude data from the specified file path.
    \item \texttt{clean\_data(data):} Cleans and validates the input data.
    \item \texttt{calculate\_statistics(data):} Computes statistical metrics such as mean, max, and min altitude.
    \item \texttt{plot\_data(data):} Generates visualizations of the altitude data.
\end{itemize}

\subsection{Example Usage}
To use the script, ensure that \texttt{altitude\_data.txt} is located in the same directory as \texttt{altitude\_data.py}. Then, run the script as follows:
\begin{verbatim}
python altitude_data.py
\end{verbatim}
The output will include statistical summaries and any generated plots.

\subsection{Future Enhancements}
Planned improvements for \texttt{altitude\_data.py} include:
\begin{itemize}
    \item Support for real-time altitude data streaming.
    \item Integration with external APIs for weather and environmental data.
    \item Enhanced visualization options, such as 3D plots.
\end{itemize}

\section{Changelog}
This section provides a summary of changes made to the Balloon project.

\subsection{Version 1.0.0 (4/14/2025)}
\subsubsection{Features}
\begin{itemize}
    \item Added \texttt{main.cpp} in the \texttt{src/} folder for core functionality.
    \item Added \texttt{testing.cpp} in the \texttt{src/} folder for altitude data processing.
    \item Added \texttt{altitude\_data.py} and \texttt{altitude\_data.txt} in the \texttt{Altitude Testing/} folder for altitude testing and data storage.
    \item Added \texttt{Adafruit\_Examples.cpp}, \texttt{Cpp\_Example.cpp}, and \texttt{Prof\_Cpp\_Example.cpp} in the \texttt{Examples/} folder for example implementations.
    \item Added \texttt{README} files in the \texttt{include/} and \texttt{lib/} folders for documentation.
    \item Added \texttt{Meeting Notes/} and \texttt{Project Notes/} in the \texttt{Notes/} folder for project-related notes.
    \item Added \texttt{changelog.txt}, \texttt{todo.txt}, and \texttt{LaTeX/} folder in the \texttt{Documentation/} folder for project documentation.
    \item Added \texttt{.vscode/} folder with configuration files for Visual Studio Code.
    \item Added \texttt{platformio.ini} for PlatformIO project configuration.
\end{itemize}

\subsubsection{Changes}
\begin{itemize}
    \item Reorganized files into a structured folder hierarchy for better project management.
    \item Modified \texttt{testing.cpp} to include functionality for reading and processing altitude data from \texttt{altitude\_data.txt}.
\end{itemize}

\subsubsection{Bug Fixes}
\begin{itemize}
    \item None.
\end{itemize}

\end{document}