\documentclass{article}
\usepackage{amsmath}
\usepackage{amssymb}



\begin{document}



\section{Changelog}
This section provides a summary of changes made to the Balloon project.



\subsection{Version 1.0.0 (4/14/2025)}
\subsubsection{Features}
\begin{itemize}
    \item Added \texttt{main.cpp} in the \texttt{src/} folder for core functionality.
    \item Added \texttt{testing.cpp} in the \texttt{src/} folder for altitude data processing.
    \item Added \texttt{altitude\_data.py} and \texttt{altitude\_data.txt} in the \texttt{Altitude Testing/} folder for altitude testing and data storage.
    \item Added \texttt{Adafruit\_Examples.cpp}, \texttt{Cpp\_Example.cpp}, and \texttt{Prof\_Cpp\_Example.cpp} in the \texttt{Examples/} folder for example implementations.
    \item Added \texttt{README} files in the \texttt{include/} and \texttt{lib/} folders for documentation.
    \item Added \texttt{Meeting Notes/} and \texttt{Project Notes/} in the \texttt{Notes/} folder for project-related notes.
    \item Added \texttt{changelog.txt}, \texttt{todo.txt}, and \texttt{LaTeX/} folder in the \texttt{Documentation/} folder for project documentation.
    \item Added \texttt{.vscode/} folder with configuration files for Visual Studio Code.
    \item Added \texttt{platformio.ini} for PlatformIO project configuration.
\end{itemize}



\subsubsection{Changes}
\begin{itemize}
    \item Reorganized files into a structured folder hierarchy for better project management.
    \item Modified \texttt{testing.cpp} to include functionality for reading and processing altitude data from \texttt{altitude\_data.txt}.
\end{itemize}



\subsubsection{Bug Fixes}
\begin{itemize}
    \item None.
\end{itemize}



\section{\texttt{main.cpp} Overview}
The \texttt{main.cpp} file is the core of the Balloon project, responsible for reading altitude data from the BMP280 sensor, processing it, and calculating key metrics such as velocity, acceleration, and jerk. It also controls the LED indicators based on the calculated jerk value.

\subsection{Functionality}
The \texttt{main.cpp} file performs the following tasks:
\begin{itemize}
    \item Initializes the BMP280 sensor and configures it for oversampling and filtering.
    \item Reads altitude data from the BMP280 sensor at a 1 Hz sampling rate.
    \item Processes the altitude data to calculate velocity, acceleration, and jerk.
    \item Outputs the calculated jerk value to the serial monitor.
    \item Controls the state of two LEDs based on the jerk value:
    \begin{itemize}
        \item Green LED: Indicates positive jerk.
        \item Red LED: Indicates negative jerk.
        \item Both LEDs off: Indicates zero jerk.
    \end{itemize}
\end{itemize}


\subsection{Key Calculations}
The highlighted section of the code calculates velocity, acceleration, and jerk using the following equations:


\subsubsection{Velocity}
Velocity is calculated as the change in altitude over time (\( \Delta t \)):
\[
v_1 = \frac{\text{altitude}_1 - \text{altitude}_0}{\Delta t}, \quad
v_2 = \frac{\text{altitude}_2 - \text{altitude}_1}{\Delta t}, \quad
v_3 = \frac{\text{altitude}_3 - \text{altitude}_2}{\Delta t}
\]



\subsubsection{Acceleration}
Acceleration is calculated as the change in velocity over time (\( \Delta t \)):
\[
a_1 = \frac{v_2 - v_1}{\Delta t}, \quad
a_2 = \frac{v_3 - v_2}{\Delta t}
\]



\subsubsection{Jerk}
Jerk is calculated as the change in acceleration over time (\( \Delta t \)):
\[
\text{jerk} = \frac{a_2 - a_1}{\Delta t}
\]



\subsection{LED Behavior}
The LEDs are controlled based on the calculated jerk value:
\begin{itemize}
    \item \textbf{Positive jerk:} The green LED is turned on, and the red LED is turned off.
    \item \textbf{Negative jerk:} The red LED is turned on, and the green LED is turned off.
    \item \textbf{Zero jerk:} Both LEDs are turned off.
\end{itemize}



\subsection{Example Output}
The \texttt{main.cpp} file outputs the current altitude and jerk value to the serial monitor in the following format:
\begin{verbatim}
Altitude: 250.00 | Jerk: 0.50
\end{verbatim}

This provides real-time feedback on the altitude and motion dynamics of the balloon.

\subsection{Future Enhancements}
Planned improvements for \texttt{main.cpp} include:
\begin{itemize}
    \item Adding error handling for sensor failures.
    \item Supporting additional sensors for enhanced data collection.
    \item Optimizing the LED control logic for energy efficiency.
\end{itemize}

\section{Balloon Project Report}

\subsection{Overview}
The Balloon Project is a program designed to collect and analyze data from a weather balloon. It utilizes sensors to measure altitude and acceleration, calculates the rate of change of acceleration (jerk) in three axes (X, Y, Z), and controls a NeoPixel LED to visually represent the jerk values. The program also logs the collected data to a CSV file and outputs it to the Serial Monitor for further analysis.

\subsection{Key Features}
\begin{itemize}
    \item Reads altitude data from a BMP280 barometric pressure sensor.
    \item Reads acceleration data from an LSM6DS33 accelerometer.
    \item Calculates jerk values for the X, Y, and Z axes.
    \item Controls a NeoPixel LED to display colors based on the direction and magnitude of jerk:
    \begin{itemize}
        \item X-axis: Red (positive) / Blue (negative)
        \item Y-axis: Green (positive) / Yellow (negative)
        \item Z-axis: Purple (positive) / White (negative)
    \end{itemize}
    \item Logs data to a CSV file on a QSPI flash filesystem.
    \item Outputs data to the Serial Monitor for real-time monitoring.
\end{itemize}

\subsection{Code Functionality}
\begin{enumerate}
    \item \textbf{Setup:} The \texttt{setup()} function initializes the I2C bus, sensors (BMP280 and LSM6DS33), NeoPixel LED, and QSPI flash filesystem. It also prepares the CSV file for data logging and prints a header to the Serial Monitor.
    \item \textbf{Main Loop:} The \texttt{loop()} function performs the following tasks:
    \begin{itemize}
        \item Reads altitude and acceleration data.
        \item Updates acceleration history arrays for each axis.
        \item Ensures sufficient data points are collected before calculating jerk.
        \item Computes jerk values using the \texttt{computeJerk()} function.
        \item Controls the NeoPixel LED based on jerk values and a predefined threshold.
        \item Logs data to the Serial Monitor and the CSV file.
    \end{itemize}
    \item \textbf{Helper Functions:}
    \begin{itemize}
        \item \texttt{scanI2CDevices()}: Scans the I2C bus for connected devices and prints their addresses.
        \item \texttt{getAltitude()}: Reads and returns the altitude from the BMP280 sensor.
        \item \texttt{getAcceleration()}: Reads acceleration data from the LSM6DS33 sensor and updates global variables for the current acceleration values.
        \item \texttt{updateAccelHistory()}: Updates the history arrays for acceleration values by shifting old values and adding new ones.
        \item \texttt{computeJerk()}: Calculates the jerk values for each axis based on the acceleration history and time interval.
    \end{itemize}
\end{enumerate}

\subsection{Data Logging}
The program logs the following data to both the Serial Monitor and a CSV file:
\begin{itemize}
    \item Time (seconds)
    \item Altitude (meters)
    \item Acceleration values for X, Y, and Z axes (m/s\textsuperscript{2})
    \item Jerk values for X, Y, and Z axes (m/s\textsuperscript{3})
\end{itemize}

\subsection{Conclusion}
This program provides a robust framework for collecting, analyzing, and visualizing data from a weather balloon. The use of sensors, real-time data logging, and visual feedback via the NeoPixel LED makes it a versatile tool for atmospheric data collection and analysis.

\end{document}